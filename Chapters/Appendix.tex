\pdfbookmark[1]{Appendix}{appendix}
% work-around to have small caps also here in the headline
\manualmark
\markboth{\spacedlowsmallcaps{Appendix}}{\spacedlowsmallcaps{Appendix}} % work-around to have small caps also
%\phantomsection 
\refstepcounter{dummy}

\chapter*{Appendix}

%thimbleby
In this appendix the final version of the ontology is given in Turtle format. It is also available to download from:\\

\verb+https://github.com/gniezen/ontologies/+

\footnotesize
\begin{verbatim}
@prefix rdf: <http://www.w3.org/1999/02/22-rdf-syntax-ns#> .
@prefix : <http://semantic.gerritniezen.com/ontologies/UserInteractionEvents.owl#> .
@prefix afn: <http://jena.hpl.hp.com/ARQ/function#> .
@prefix spin: <http://spinrdf.org/spin#> .
@prefix sp: <http://spinrdf.org/sp#> .
@prefix owl: <http://www.w3.org/2002/07/owl#> .
@prefix xsd: <http://www.w3.org/2001/XMLSchema#> .
@prefix rdfs: <http://www.w3.org/2000/01/rdf-schema#> .
@prefix spl: <http://spinrdf.org/spl#> .

<http://semantic.gerritniezen.com/ontologies/UserInteractionEvents.owl>
    spin:imports <http://topbraid.org/spin/owlrl-all> ;
    a owl:Ontology ;
    owl:imports <http://spinrdf.org/spin> ;
    owl:versionInfo "Created with TopBraid Composer"^^xsd:string .

:AdjustLevel
    a :Functionality .
:AdjustLevelEvent
    a owl:Class ;
    rdfs:subClassOf :InteractionEvent .
:Alarm
    a :Functionality .
:AlarmAlertEvent
    a owl:Class ;
    rdfs:subClassOf :AlarmEvent .
:AlarmEndEvent
    a owl:Class ;
    rdfs:subClassOf :AlarmEvent .
:AlarmEvent
    a owl:Class ;
    rdfs:subClassOf :InteractionEvent .
:AlarmRemoveEvent
    a owl:Class ;
    rdfs:subClassOf :AlarmEvent .
:AlarmSetEvent
    a owl:Class ;
    rdfs:subClassOf :AlarmEvent, :SetEvent ;
    owl:equivalentClass [
        a owl:Class ;
        owl:intersectionOf (:SetEvent
            [
                a owl:Restriction ;
                owl:allValuesFrom xsd:dateTime ;
                owl:onProperty :dataValue
            ]
            [
                a owl:Restriction ;
                owl:cardinality "1"^^xsd:nonNegativeInteger ;
                owl:onProperty :dataValue
            ]
        )
    ] .
:Audio
    a :MediaType .
:Binary
    a :RangeMeasure ;
    rdfs:comment "True/False, 0 or 1"^^xsd:string .
:Bridge
    a owl:Class ;
    rdfs:subClassOf :SmartObject ;
    owl:equivalentClass [
        a owl:Class ;
        owl:intersectionOf (:SmartObject
            :Sink
            :Source
        )
    ] .
:CueEvent
    a owl:Class ;
    rdfs:subClassOf :MediaPlayerEvent ;
    owl:disjointWith :PlayEvent, :StopEvent .
:Double
    a :RangeMeasure ;
    rdfs:comment "Capable of representing values from 0-99 (double digits)" .
:Event
    a owl:Class ;
    rdfs:subClassOf owl:Thing .
:FeedbackEvent
    a owl:Class ;
    rdfs:subClassOf :PreviewEvent .
:Functionality
    a owl:Class ;
    rdfs:subClassOf owl:Thing .
:IDType
    a owl:Class ;
    rdfs:subClassOf owl:Thing .
:IPAddress
    a :IDType .
:Identification
    a owl:Class ;
    rdfs:subClassOf owl:Thing .
:IncreaseLevelEvent
    a owl:Class ;
    rdfs:subClassOf :AdjustLevelEvent ;
    owl:equivalentClass [
        a owl:Class ;
        owl:intersectionOf (:AdjustLevelEvent
            [
                a owl:Restriction ;
                owl:onProperty :duration ;
                owl:someValuesFrom xsd:integer
            ]
        )
    ] .
:IndicatorEvent
    a owl:Class ;
    rdfs:subClassOf :PreviewEvent .
:InteractionEvent
    a owl:Class ;
    rdfs:subClassOf :Event .
:InteractionPrimitive
    a owl:Class ;
    rdfs:comment "Smallest addressable interaction element 
that has a meaningful relation to the interaction itself"^^xsd:string ;
    rdfs:subClassOf owl:Thing .
:MediaPath
    a owl:Class ;
    rdfs:subClassOf owl:Thing .
:MediaPlayerEvent
    a owl:Class ;
    rdfs:subClassOf :InteractionEvent .
:MediaType
    a owl:Class ;
    rdfs:subClassOf owl:Thing .
:Music
    a :Functionality .
:PlayEvent
    a owl:Class ;
    rdfs:subClassOf :MediaPlayerEvent ;
    owl:disjointWith :CueEvent, :StopEvent .
:PreviewEvent
    a owl:Class ;
    rdfs:subClassOf :InteractionEvent .
:RFID_Mifare
    a :IDType .
:RGBValues
    a :MediaType .
:RangeMeasure
    a owl:Class ;
    rdfs:subClassOf owl:Thing .
:Real
    a :RangeMeasure ;
    rdfs:comment "Capable of representing more than 1000 discrete values (infinite)" .
:SemanticTransformer
    spin:rule [
        sp:templates ([
                sp:object spin:_this ;
                sp:predicate :connectedTo ;
                sp:subject _:A2
            ]
            [
                sp:object _:A3 ;
                sp:predicate :connectedTo ;
                sp:subject spin:_this
            ]
        ) ;
        sp:where ([
                sp:object spin:_this ;
                sp:predicate :canConnectTo ;
                sp:subject _:A2
            ]
            [
                sp:object _:A3 ;
                sp:predicate :canConnectTo ;
                sp:subject spin:_this
            ]
            [
                sp:object _:A3 ;
                sp:predicate :connectedTo ;
                sp:subject _:A2
            ]
        ) ;
        a sp:Construct
    ], [
        sp:templates ([
                sp:object _:A1 ;
                sp:predicate spin:_this ;
                sp:subject _:A0
            ]
        ) ;
        sp:where ([
                sp:object _:A1 ;
                sp:predicate :canConnectTo ;
                sp:subject _:A0
            ]
        ) ;
        a sp:Construct
    ] ;
    a owl:Class ;
    rdfs:subClassOf owl:Thing ;
    owl:disjointWith :SmartObject ;
    owl:equivalentClass [
        a owl:Class ;
        owl:intersectionOf (:SemanticTransformer
            [
                a owl:Restriction ;
                owl:cardinality "1"^^xsd:nonNegativeInteger ;
                owl:onProperty :functionalitySource
            ]
        )
    ], [
        a owl:Class ;
        owl:intersectionOf ([
                a owl:Restriction ;
                owl:onProperty :canAcceptMediaTypeFrom ;
                owl:someValuesFrom :SmartObject
            ]
            [
                a owl:Restriction ;
                owl:onProperty :convertsMediaType ;
                owl:someValuesFrom :SmartObject
            ]
        )
    ] .
:SetEvent
    a owl:Class ;
    rdfs:subClassOf :InteractionEvent .
:Single
    a :RangeMeasure ;
    rdfs:comment "Capable of representing values from 0-9 (single digits)" .
:Sink
    a owl:Class ;
    rdfs:subClassOf :SmartObject ;
    owl:equivalentClass [
        a owl:Class ;
        owl:intersectionOf (:SmartObject
            [
                a owl:Restriction ;
                owl:onProperty :functionalitySink ;
                owl:someValuesFrom :Functionality
            ]
        )
    ] .
:SmartObject
    spin:rule [
        sp:templates ([
                sp:object _:A5 ;
                sp:predicate _:A4 ;
                sp:subject spin:_this
            ]
        ) ;
        sp:where ([
                sp:object _:A4 ;
                sp:predicate :functionalitySource ;
                sp:subject spin:_this
            ]
            [
                sp:object _:A4 ;
                sp:predicate :functionalitySink ;
                sp:subject _:A5
            ]
        ) ;
        a sp:Construct
    ], [
        sp:templates ([
                sp:object _:A6 ;
                sp:predicate :hasLastEvent ;
                sp:subject spin:_this
            ]
        ) ;
        sp:where ([
                sp:expression [
                    sp:arg1 spin:_this ;
                    a :getMaxDateRsc
                ] ;
                sp:variable _:A6 ;
                a sp:Bind
            ]
        ) ;
        a sp:Construct
    ], [
        sp:templates ([
                sp:object _:A8 ;
                sp:predicate :hasRFIDTag ;
                sp:subject spin:_this
            ]
        ) ;
        sp:where ([
                sp:object :RFID_Mifare ;
                sp:predicate :ofIDType ;
                sp:subject _:A7
            ]
            [
                sp:object _:A7 ;
                sp:predicate :hasIdentification ;
                sp:subject spin:_this
            ]
            [
                sp:object _:A8 ;
                sp:predicate :idValue ;
                sp:subject _:A7
            ]
        ) ;
        a sp:Construct
    ] ;
    a owl:Class ;
    rdfs:subClassOf owl:Thing .
:Source
    a owl:Class ;
    rdfs:subClassOf :SmartObject ;
    owl:equivalentClass [
        a owl:Class ;
        owl:intersectionOf (:SmartObject
            [
                a owl:Restriction ;
                owl:onProperty :functionalitySource ;
                owl:someValuesFrom :Functionality
            ]
        )
    ] .
:StopEvent
    a owl:Class ;
    rdfs:subClassOf :MediaPlayerEvent ;
    owl:disjointWith :CueEvent, :PlayEvent .
:SystemEvent
    a owl:Class ;
    rdfs:subClassOf :Event .
:TCPIPObject
    a owl:Class ;
    rdfs:subClassOf :SmartObject ;
    owl:equivalentClass [
        a owl:Restriction ;
        owl:hasValue :IPAddress ;
        owl:onProperty :hasIDType
    ], [
        a owl:Restriction ;
        owl:hasValue true ;
        owl:onProperty :communicatesByTCPIP
    ] .
:TimeSetEvent
    a owl:Class ;
    rdfs:subClassOf :SystemEvent .
:Triple
    a :RangeMeasure ;
    rdfs:comment "Capable of representing values from 0-999 (triple digits)" .
:acceptsMediaType
    a owl:ObjectProperty .
:canAcceptMediaTypeFrom
    a owl:ObjectProperty ;
    owl:inverseOf :convertsMediaType .
:canConnectTo
    a owl:IrreflexiveProperty, owl:ObjectProperty ;
    owl:propertyChainAxiom (:functionalitySource
        :functionalitySink
    ) .
:canIndirectlyConnectTo
    a owl:ObjectProperty ;
    rdfs:subPropertyOf :canConnectTo ;
    owl:propertyChainAxiom (:canConnectTo
        :canConnectTo
    ) .
:communicatesByTCPIP
    a owl:DatatypeProperty ;
    rdfs:domain :SmartObject ;
    rdfs:range xsd:boolean .
:connectedFrom
    a owl:ObjectProperty ;
    owl:inverseOf :connectedTo .
:connectedTo
    a owl:IrreflexiveProperty, owl:ObjectProperty ;
    rdfs:domain :Source ;
    rdfs:range :Sink .
:convertsMediaType
    a owl:IrreflexiveProperty, owl:ObjectProperty ;
    owl:propertyChainAxiom (:transmitsMediaType
        :isAcceptedMediaTypeOf
    ) .
:cueAt
    a owl:DatatypeProperty ;
    rdfs:comment "Cue at time (in milliseconds)"^^xsd:string ;
    rdfs:domain :CueEvent ;
    rdfs:range xsd:integer .
:currentDateTime
    spin:body [
        sp:resultVariables (_:A9
        ) ;
        sp:where ([
                sp:expression [
                    a afn:now
                ] ;
                sp:variable _:A9 ;
                a sp:Bind
            ]
        ) ;
        a sp:Select
    ] ;
    a spin:MagicProperty ;
    rdfs:subClassOf spin:MagicProperties .
:dataValue
    a owl:DatatypeProperty ;
    rdfs:range xsd:anySimpleType .
:duration
    a owl:DatatypeProperty .
:functionalitySink
    a owl:ObjectProperty .
:functionalitySource
    a owl:ObjectProperty .
:generatedBy
    a owl:ObjectProperty ;
    rdfs:range :SmartObject .
:getMaxDateRsc
    spin:body [
        sp:limit "1"^^xsd:long ;
        sp:orderBy ([
                sp:expression _:A11 ;
                a sp:Desc
            ]
        ) ;
        sp:resultVariables (_:A10
        ) ;
        sp:where ([
                sp:object spin:_arg1 ;
                sp:predicate :generatedBy ;
                sp:subject _:A10
            ]
            [
                sp:object _:A11 ;
                sp:predicate :inXSDDateTime ;
                sp:subject _:A10
            ]
        ) ;
        a sp:Select
    ] ;
    spin:constraint [
        spl:predicate sp:arg1 ;
        a spl:Argument ;
        rdfs:comment "Smart object that generated the interaction event"^^xsd:string
    ] ;
    a spin:Function ;
    rdfs:subClassOf spin:Functions .
:hasIDType
    a owl:ObjectProperty ;
    owl:propertyChainAxiom (:hasIdentification
        :ofIDType
    ) .
:hasIdentification
    a owl:ObjectProperty ;
    rdfs:range :Identification .
:hasInteractionPrimitive
    a owl:ObjectProperty .
:hasLastEvent
    a owl:ObjectProperty .
:hasMediaPath
    a owl:ObjectProperty .
:hasRFIDTag
    a owl:DatatypeProperty .
:hasRangeMeasure
    a owl:ObjectProperty ;
    rdfs:domain :InteractionPrimitive ;
    rdfs:range :RangeMeasure .
:idValue
    a owl:DatatypeProperty ;
    rdfs:domain :Identification .
:identificationOf
    a owl:ObjectProperty ;
    owl:inverseOf :hasIdentification .
:inXSDDateTime
    a owl:DatatypeProperty ;
    rdfs:domain :Event ;
    rdfs:range xsd:dateTime .
:indirectlyConnectedTo
    a owl:ObjectProperty ;
    rdfs:subPropertyOf :connectedTo .
:isAcceptedMediaTypeOf
    a owl:ObjectProperty ;
    owl:inverseOf :acceptsMediaType .
:isFunctionalityOfSink
    a owl:ObjectProperty ;
    owl:inverseOf :functionalitySink .
:isFunctionalityOfSource
    a owl:ObjectProperty ;
    owl:inverseOf :functionalitySource .
:isInteractionPrimitiveOf
    a owl:ObjectProperty ;
    owl:inverseOf :hasInteractionPrimitive .
:launchesEvent
    a owl:ObjectProperty ;
    owl:inverseOf :generatedBy .
:mediaOriginator
    a owl:ObjectProperty .
:mediaPathOf
    a owl:ObjectProperty ;
    owl:inverseOf :hasMediaPath .
:mediaSourceSO
    a owl:ObjectProperty .
:ofIDType
    a owl:ObjectProperty ;
    rdfs:domain :Identification ;
    rdfs:range :IDType .
:tempConnectedTo
    a owl:ObjectProperty .
:transmitsMediaType
    a owl:ObjectProperty .
_:A0
    sp:varName "source"^^xsd:string .
_:A1
    sp:varName "sink"^^xsd:string .
_:A10
    sp:varName "lastEvent"^^xsd:string .
_:A11
    sp:varName "last"^^xsd:string .
_:A2
    sp:varName "source"^^xsd:string .
_:A3
    sp:varName "sink"^^xsd:string .
_:A4
    sp:varName "functionality"^^xsd:string .
_:A5
    sp:varName "sink"^^xsd:string .
_:A6
    sp:varName "lastEvent"^^xsd:string .
_:A7
    sp:varName "id"^^xsd:string .
_:A8
    sp:varName "tag"^^xsd:string .
_:A9
    sp:varName "datetime"^^xsd:string .
\end{verbatim}
\normalsize

% Replace below with ontology listing
%
% In Appendix A various tips \& tricks are listed for working with the applications and tools described in the thesis.
% 
% \section{Working with Smart-M3}
% 
% The following was performed in \texttt{smart-m3\_v0.9.2-beta} after compiling and installing everything according to the Smart-M3 Setup Guide.\\
% 
% Run:
% \begin{itemize}
% 	\item \texttt{sibd} (SIB daemon)
% 	\item \texttt{sib-tcp} (TCP connector to SIB daemon)
% \end{itemize}
% 
% If both are installed correctly, you should be able to run them from the command-line without any problems, and they will not produce any output directly after starting up.
% 
% The following Python scripts are available in \texttt{m3\_sofia\_1305.zip} on the SOFIA project website, WP5 M3 Implementation\footnote{http://www.sofia-project.eu/node/226}.
% 
% When running \texttt{basic\_test.py}, you have to enter the following parameters:
% 
% \begin{verbatim}
% 	Manual Discovery. Enter details:
% 	SmartSpace name       >test  (or any other name for the smart space)
% 	SmartSpace IP Address >127.0.0.1                          	
% 	SmartSpace Port       >10010
% \end{verbatim}
% 
% \texttt{basic\_test.py} then produces the following results:
% 
% \begin{verbatim}
% test 127.0.0.1 10010
% ('test', (<class smart_m3.Node.TCPConnector at 
% 			0x1fae230>, ('127.0.0.1', 10010)))
% --- Member of SS: ['test']
% RDF Subscribe initial result: []
% WQL values subscribe initial result: []
% Subscription:
% Added: ((u'x1', u'lives', u'Espoo'), True)
% Added: ((u'x2', u'lives', u'Espoo'), True)
% Subscription:
% Added: x1 drinks (u'beer', True)
% Querying what is being drunk
% Subscription:
% Added: x1 drinks (u'wine', True)
% Removed: x1 drinks (u'beer', True)
% QUERY: Got triple(s):  ((u'x1', u'drinks', u'wine'), True)
% Querying: type of x1
% 
% Querying: is person a subtype of thing
% QUERY: person is a subtype of thing is:  True
% 
% Querying: x1 and x2 related via 'knows' property
% QUERY: x1 knows x2 is True
% 
% Querying: is x1 a person
% QUERY: 'x1 is a person' is  True
% 
% Querying: which persons live in Espoo
% QUERY: persons living in Espoo: [(u'x1', False), (u'x2', False)]
% Querying: all triples
% 
% ...
% 
% Subscription:
% Removed: ((u'x2', u'lives', u'Espoo'), True)
% Unsubscribing RDF subscription
% Unsubscribing WQL subscription
% Left smart space
% \end{verbatim}
% 
% While \texttt{basic\_test.py} is running, \texttt{sib-tcp} remains silent, but \texttt{sibd} produces the following results:
% 
% \begin{verbatim}
% 	SUBSCRIBE: subscription SIB Tester-35053592 finished
% 	UNSUBSCRIBE: Sent unsub cnf for sub id SIB Tester-35053592
% 	SUBSCRIBE: subscription SIB Tester-35053592 finished
% 	UNSUBSCRIBE: Sent unsub cnf for sub id SIB Tester-35053592
% \end{verbatim}
% 
% Using \texttt{sofia\_release\_1305} from the SOFIA website is also possible by running \verb|python SIB.py X|, where X is the name of the smart space. You may then connect to the smart space using the the \texttt{basic\_test.py} program supplied.
% \texttt{explorer.py} is a Qt-based viewer that may be used to view the triples in the triple store.
% 
% 
% \section{Working with TopBraid Composer}
% 
% To prevent the OWL 2 RL classes from showing up in the Classes view your ontology when using TopBraid Composer, use 
% 
% \begin{minted}{turtle}
% spin:imports <http://topbraid.org/spin/owlrl-all>
% \end{minted}
% 
% to import them instead of \texttt{owl:imports}. This also prevents unusual syntax errors when performing constraint checking. However, it is important to still import the SPIN library using 
% 
% \begin{minted}{turtle}
% owl:imports <http://spinrdf.org/spin>
% \end{minted}
% 
% instead of \texttt{spin:imports}, otherwise the SPIN rules cannot be resolved properly.\\
% 
% To view all SPIN rules,use\\
% 
% Model~$\Rightarrow$~Display SPIN rules and constraints
% 
% 
% 
% \section{Setting up the environment}
% 
% The last version we tested was ADK-SIB v.2.0.5 (from sg1.esilab.org/sofia) with Eclipse Helios v.3.6.2.
% 
% When we tested v.2.0.8, it broke SSAP compatibility with the Smart-M3 implementation (used for Python KPs).
% 
% Eclipse Indigo is incompatible with the OSGi implementation of the ADK-SIB, where the \texttt{registerService} function now requires a \texttt{Dictionary} instead of \texttt{Properties}.
% 
% \section{Loading local version of an ontology when available}
% 
% When using the Jena API to import an ontology on the web, you can specify a local version on your computer to be used instead. This is quite useful when developing your own ontology.
% 
% First you need to create a file that maps your locations. Mine is called \texttt{location-mappings.ttl}: 
% 
% \begin{minted}{turtle}
% @prefix lm: <http://jena.hpl.hp.com/2004/08/location-mapping#> . 
% 
% [] lm:mapping 
% [ lm:name 
% "http://sofia.gotdns.com/ontologies/SemanticConnections.owl" ; 
% lm:altName "file:SemanticConnections.owl"  ] , 
% [ lm:name 
% "http://sofia.gotdns.com/ontologies/SemanticInteraction.owl" ; 
%   lm:altName "file:SemanticInteraction.owl"  ] . 
% \end{minted}
% 
% Then you define it in your code before loading the ontology: 
% 
% \begin{minted}{java}
% import com.hp.hpl.jena.util.LocationMapper; 
% 
% LocationMapper lm= new LocationMapper("location-mapping.ttl"); 
% LocationMapper.setGlobalLocationMapper(lm); 
% FileManager.get().setLocationMapper(lm);
% \end{minted}
% 
% If all goes well, it will load the local versions of specified imported ontologies.
% 
% \section{Using the SIB}
% 
% \subsection{Retrieving the machine's IP address}
% The standard SOFIA version of the ADK-SIB uses the Java \texttt{get\-LocalHost()} function to get the local IP address of the machine, but this does not always work correctly. \texttt{getLocalHost()} uses the machine's network name to retrieve the IP address. In our case, the network name \texttt{ID00713} was mapped to the VNC IP address, not the public IP address. As \texttt{getLocalHost()} ignores the \texttt{hosts} file on the machine, a version of the ADK-SIB was created that uses the Java \texttt{getByName()} function instead.
% 
% \subsection{Packaging the SIB in an OSGi bundle}
% The following steps were followed to create an OSGI bundle of the modified ADK-SIB:
% 
% \begin{itemize}
% 	\item Get the latest version of the SOFIA ADK-SIB source code, e.g. we used v.2.0.5 in \texttt{eu.sofia.sib.osgi} from \texttt{esilab.org}.
% 	\item Add \texttt{log4j.properties} to \texttt{build.properties} to ensure that a log file is generated.
% 	\item For SPIN, add the following .jar files and also put them in \texttt{MANIFEST.MF} (we used version 1.2.0):
% 	\begin{itemize}
% 		\item spin1.2.0
% 		\item spinbase-1.2.0
% 		\item spin.functions
% 		\item arq-2.8.7
% 		\item jena-2.8.7
% 	\end{itemize}
% 	\item Add the above to \texttt{build.properties}, as well as \texttt{Properties}~$\Rightarrow$~\texttt{Java Build Path}
% 	\item Remove the \texttt{extends} from the \texttt{ResultSetFormatter} function (this causes a problem with ARQ)
% 	\item Also import \texttt{eu.sofia.adk.gateway.tcpip} and modify the Run Configuration to use the latest bundles
% 	\item Perform Eclipse Update to get new bundles for SOFIA ADK
% 	
% \end{itemize}
% 
% 
% \section{Useful SPARQL queries}
% 
% To get all datatype instances: %17/10 notebook 2
% \begin{minted}{sparql}
% SELECT DISTINCT ?s ?p ?o
% WHERE {
% 	?s ?p ?o .
% 	FILTER (isLiteral(?o)) .
% }
% \end{minted}


% TO HERE



% Possible TODO: Notebook 2 24/04 (Serving ontology at different URI)
% Possible TODO: Comparing TikZ to GraphViz
% Possible TODO: Device documentation from ~/code/semanticconnections/docs
% Possible TODO: Smart-M3 SIB installation 06/09/10



% - Tools: 
% 	- ssls (See README)
% 		.sslslogin and .ssls
% 		Reported various bugs that were fixed (see e.g. 31/01/11)
% 	End of notebook1:
% 	- d-bus
% 	- git
% 	- svn
% 	-
% 
% - System setup
% 	07/06/11 (Location of source code, also see README for Aly's deliverable)
% 
% - ADK-SIB
% 	To update deployable .jar (19/01/11)
% 	Does not support triple-level queries without at least one wildcard, i.e. cannot be used to test if a certain triple exists in the triple store
% 	Updated Jena to 2.6.4 to be compatible with SPIN 1.2.0 (uses Jena ARQ-2.8.7 as SPARQL engine)
% 		- Requires \texttt{extends} to be removed from \texttt{ResultSetFormatter} constructor in ADK-SIB, which was still valid with ARQ-2.8.2
% 	Packaging ADK-SIB
% 		log4j.properties should be added to build.properties
% 		Reinstalling bundles
% 			stop <bundle>
% 			uninstall <bundle>
% 			install file://home/gerrit/...
% 			ss (to view bundles)
% 			start <bundle>
% 				
% - Developing with Android
% 	04/04/11, 02/05/11
% 	Python KPI: getprotobyname in Node.py is unsupported, but not needed