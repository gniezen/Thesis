\chapter{Appendix}

In Appendix A various tips \& tricks are listed for working with the applications and tools described in the thesis.

\section{Working with Smart-M3}

The following was performed in \texttt{smart-m3\_v0.9.2-beta} after compiling and installing everything according to the Smart-M3 Setup Guide.\\

Run:
\begin{itemize}
	\item \texttt{sibd} (SIB daemon)
	\item \texttt{sib-tcp} (TCP connector to SIB daemon)
\end{itemize}

If both are installed correctly, you should be able to run them from the command-line without any problems, and they will not produce any output directly after starting up.

The following Python scripts are available in \texttt{m3\_sofia\_1305.zip} on the SOFIA project website, WP5 M3 Implementation\footnote{http://www.sofia-project.eu/node/226}.

When running \texttt{basic\_test.py}, you have to enter the following parameters:

\begin{verbatim}
	Manual Discovery. Enter details:
	SmartSpace name       >test  (or any other name for the smart space)
	SmartSpace IP Address >127.0.0.1                          	
	SmartSpace Port       >10010
\end{verbatim}

\texttt{basic\_test.py} then produces the following results:

\begin{verbatim}
test 127.0.0.1 10010
('test', (<class smart_m3.Node.TCPConnector at 
			0x1fae230>, ('127.0.0.1', 10010)))
--- Member of SS: ['test']
RDF Subscribe initial result: []
WQL values subscribe initial result: []
Subscription:
Added: ((u'x1', u'lives', u'Espoo'), True)
Added: ((u'x2', u'lives', u'Espoo'), True)
Subscription:
Added: x1 drinks (u'beer', True)
Querying what is being drunk
Subscription:
Added: x1 drinks (u'wine', True)
Removed: x1 drinks (u'beer', True)
QUERY: Got triple(s):  ((u'x1', u'drinks', u'wine'), True)
Querying: type of x1

Querying: is person a subtype of thing
QUERY: person is a subtype of thing is:  True

Querying: x1 and x2 related via 'knows' property
QUERY: x1 knows x2 is True

Querying: is x1 a person
QUERY: 'x1 is a person' is  True

Querying: which persons live in Espoo
QUERY: persons living in Espoo: [(u'x1', False), (u'x2', False)]
Querying: all triples

...

Subscription:
Removed: ((u'x2', u'lives', u'Espoo'), True)
Unsubscribing RDF subscription
Unsubscribing WQL subscription
Left smart space
\end{verbatim}

While \texttt{basic\_test.py} is running, \texttt{sib-tcp} remains silent, but \texttt{sibd} produces the following results:

\begin{verbatim}
	SUBSCRIBE: subscription SIB Tester-35053592 finished
	UNSUBSCRIBE: Sent unsub cnf for sub id SIB Tester-35053592
	SUBSCRIBE: subscription SIB Tester-35053592 finished
	UNSUBSCRIBE: Sent unsub cnf for sub id SIB Tester-35053592
\end{verbatim}

Using \texttt{sofia\_release\_1305} from the SOFIA website is also possible by running \verb|python SIB.py X|, where X is the name of the smart space. You may then connect to the smart space using the the \texttt{basic\_test.py} program supplied.
\texttt{explorer.py} is a Qt-based viewer that may be used to view the triples in the triple store.


\section{Working with TopBraid Composer}

To prevent the OWL 2 RL classes from showing up in the Classes view your ontology when using TopBraid Composer, use 

\begin{minted}{turtle}
spin:imports <http://topbraid.org/spin/owlrl-all>
\end{minted}

to import them instead of \texttt{owl:imports}. This also prevents unusual syntax errors when performing constraint checking. However, it is important to still import the SPIN library using 

\begin{minted}{turtle}
owl:imports <http://spinrdf.org/spin>
\end{minted}

instead of \texttt{spin:imports}, otherwise the SPIN rules cannot be resolved properly.\\

To view all SPIN rules,use\\

Model~$\Rightarrow$~Display SPIN rules and constraints



\section{Setting up the environment}

The last version we tested was ADK-SIB v.2.0.5 (from sg1.esilab.org/sofia) with Eclipse Helios v.3.6.2.

When we tested v.2.0.8, it broke SSAP compatibility with the Smart-M3 implementation (used for Python KPs).

Eclipse Indigo is incompatible with the OSGi implementation of the ADK-SIB, where the \texttt{registerService} function now requires a \texttt{Dictionary} instead of \texttt{Properties}.

\section{Loading local version of an ontology when available}

When using the Jena API to import an ontology on the web, you can specify a local version on your computer to be used instead. This is quite useful when developing your own ontology.

First you need to create a file that maps your locations. Mine is called \texttt{location-mappings.ttl}: 

\begin{minted}{turtle}
@prefix lm: <http://jena.hpl.hp.com/2004/08/location-mapping#> . 

[] lm:mapping 
[ lm:name 
"http://sofia.gotdns.com/ontologies/SemanticConnections.owl" ; 
lm:altName "file:SemanticConnections.owl"  ] , 
[ lm:name 
"http://sofia.gotdns.com/ontologies/SemanticInteraction.owl" ; 
  lm:altName "file:SemanticInteraction.owl"  ] . 
\end{minted}

Then you define it in your code before loading the ontology: 

\begin{minted}{java}
import com.hp.hpl.jena.util.LocationMapper; 

LocationMapper lm= new LocationMapper("location-mapping.ttl"); 
LocationMapper.setGlobalLocationMapper(lm); 
FileManager.get().setLocationMapper(lm);
\end{minted}

If all goes well, it will load the local versions of specified imported ontologies.

\section{Using the SIB}

\subsection{Retrieving the machine's IP address}
The standard SOFIA version of the ADK-SIB uses the Java \texttt{get\-LocalHost()} function to get the local IP address of the machine, but this does not always work correctly. \texttt{getLocalHost()} uses the machine's network name to retrieve the IP address. In our case, the network name \texttt{ID00713} was mapped to the VNC IP address, not the public IP address. As \texttt{getLocalHost()} ignores the \texttt{hosts} file on the machine, a version of the ADK-SIB was created that uses the Java \texttt{getByName()} function instead.

\subsection{Packaging the SIB in an OSGi bundle}
The following steps were followed to create an OSGI bundle of the modified ADK-SIB:

\begin{itemize}
	\item Get the latest version of the SOFIA ADK-SIB source code, e.g. we used v.2.0.5 in \texttt{eu.sofia.sib.osgi} from \texttt{esilab.org}.
	\item Add \texttt{log4j.properties} to \texttt{build.properties} to ensure that a log file is generated.
	\item For SPIN, add the following .jar files and also put them in \texttt{MANIFEST.MF} (we used version 1.2.0):
	\begin{itemize}
		\item spin1.2.0
		\item spinbase-1.2.0
		\item spin.functions
		\item arq-2.8.7
		\item jena-2.8.7
	\end{itemize}
	\item Add the above to \texttt{build.properties}, as well as \texttt{Properties}~$\Rightarrow$~\texttt{Java Build Path}
	\item Remove the \texttt{extends} from the \texttt{ResultSetFormatter} function (this causes a problem with ARQ)
	\item Also import \texttt{eu.sofia.adk.gateway.tcpip} and modify the Run Configuration to use the latest bundles
	\item Perform Eclipse Update to get new bundles for SOFIA ADK
	
\end{itemize}


\section{Useful SPARQL queries}

To get all datatype instances: %17/10 notebook 2
\begin{minted}{sparql}
SELECT DISTINCT ?s ?p ?o
WHERE {
	?s ?p ?o .
	FILTER (isLiteral(?o)) .
}
\end{minted}






% Possible TODO: Notebook 2 24/04 (Serving ontology at different URI)
% Possible TODO: Comparing TikZ to GraphViz
% Possible TODO: Device documentation from ~/code/semanticconnections/docs
% Possible TODO: Smart-M3 SIB installation 06/09/10



% - Tools: 
% 	- ssls (See README)
% 		.sslslogin and .ssls
% 		Reported various bugs that were fixed (see e.g. 31/01/11)
% 	End of notebook1:
% 	- d-bus
% 	- git
% 	- svn
% 	-
% 
% - System setup
% 	07/06/11 (Location of source code, also see README for Aly's deliverable)
% 
% - ADK-SIB
% 	To update deployable .jar (19/01/11)
% 	Does not support triple-level queries without at least one wildcard, i.e. cannot be used to test if a certain triple exists in the triple store
% 	Updated Jena to 2.6.4 to be compatible with SPIN 1.2.0 (uses Jena ARQ-2.8.7 as SPARQL engine)
% 		- Requires \texttt{extends} to be removed from \texttt{ResultSetFormatter} constructor in ADK-SIB, which was still valid with ARQ-2.8.2
% 	Packaging ADK-SIB
% 		log4j.properties should be added to build.properties
% 		Reinstalling bundles
% 			stop <bundle>
% 			uninstall <bundle>
% 			install file://home/gerrit/...
% 			ss (to view bundles)
% 			start <bundle>
% 				
% - Developing with Android
% 	04/04/11, 02/05/11
% 	Python KPI: getprotobyname in Node.py is unsupported, but not needed