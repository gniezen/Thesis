\cleardoublepage
%\pagenumbering{gobble}
\pagestyle{empty}

%*******************************************************
% Abstract
%*******************************************************
%\renewcommand{\abstractname}{Abstract}
\pdfbookmark[1]{Summary}{Summary}
\begingroup
\let\clearpage\relax
\let\cleardoublepage\relax
\let\cleardoublepage\relax

\chapter*{Summary}

\begingroup
    \spacedallcaps{\myTitle} \\ 
\endgroup

\begingroup
\noindent
\spacedlowsmallcaps{\mySubtitle}\\ \bigskip
\endgroup

The thesis describes the design and development of an ontology and software framework to support user interaction in ubiquitous computing scenarios. The key goal of ubiquitous computing is ``serendipitous interoperability'', where devices that were not necessarily designed to work together should be able to discover each other's functionality and be able to make use of it. Future ubiquitous computing scenarios involve hundreds of devices. Therefore, anticipating all the different types of devices and usage scenarios a priori is an unmanageable task.

An iterative approach was followed during the design process, with three design iterations documented in the thesis. The work was done in close cooperation with designers and other project partners, in order to elicit requirements and maintain a more holistic view of the various application areas.
 
The thesis describes an interaction model that shows the various concepts that are involved in user interaction in a smart space, including how these concepts work together. Based in the interaction model, a theory of semantic connections is introduced that focuses on the meaning of the connections between the different entities in a smart environment.

Ontologies are formal representations of concepts in a domain of interest and the relationships between these concepts. They are used to enable the exchange of information without requiring up-front standardisation. The ontology described in the thesis helps developers to focus on modelling the interaction capabilities of smart objects and inferring the possible connections between these objects, making it easier to build smart objects and enable device interoperability on a semantic level.

Rather than just describing the low-level hardware input event that triggered an action, interaction events in the ontology are modelled as high-level input actions which report the intent of the user's action directly. This allows developers to write software that respond to these high-level events, without having to support every kind of device that could have generated that event. The event hierarchy can be inferred using semantic reasoning.

The software architecture implements the publish/subscribe messaging paradigm, enabling smart objects to subscribe to changes in data, represented in triple form, and be notified every time these triples are updated, added or removed. Semantic reasoning is performed on an information broker, simplifying the implementation on the smart objects.

A pilot deployment, composed of heterogeneous smart objects designed and manufactured by a range of companies and institutions, was used to validate the design. A performance evaluation was performed, where the results indicated acceptable response times for a networked user interface. A usability analysis of the ontology and system implementation was performed using a developer questionnaire based on an existing usability framework. Various ontology design patterns were identified during the course of the design, and are documented in the thesis.

The resulting design artefact is an ontology for user interaction with devices in a smart environment, where devices are able to share interaction events and make use of each other's functionality.


\vfill

\pdfbookmark[1]{Samenvatting}{Samenvatting}
\chapter*{Samenvatting}

Dit proefschrift beschrijft het ontwerp en de ontwikkeling van een ontologie en software raamwerk ter ondersteuning van gebruikersinteractie in ubiquitous computing scenario's. De kern van ubiquitous computing is ``serendipitous interoperability'', waarbij apparatuur die niet noodzakelijkerwijs ontworpen is om samen te werken, in staat zou moeten zijn om elkaars functionaliteit te ontdekken en te gebruiken. In toekomstige ubiquitous computing scenario's zijn honderden apparaten betrokken. Daarom is het op voorhand overzien van de verschillende types apparaten en gebruikersscenario's een onhanteerbare opgave. 

In het ontwerptraject is een iteratieve aanpak toegepast, waarvan er drie ontwerpiteraties in het proefschrift gedocumenteerd staan. Het werk is gedaan in nauwe samenwerking met ontwerpers en andere projectpartners, om ontwerprichtlijnen te extraheren en een holistische kijk op de verschillende toepassingsgebieden te waarborgen. 

Het proefschrift beschrijft een interactiemodel dat de verschillende concepten met betrekking tot gebruikersinteracties in een intelligente omgeving laat zien, inclusief hoe deze concepten onderling samenwerken. Gebaseerd op het interactiemodel, is een theorie ge\"introduceerd die zich richt op de betekenis van de verbindingen tussen de verschillende entiteiten in een intelligente omgeving. 

Ontologie\"en zijn formele beschrijvingen van concepten in een bepaald interessegebied en de verhoudingen tussen deze concepten. Ze worden gebruikt om informatie uit te wisselen zonder een voorafgaande standaardisatie te vereisen. De ontologie beschreven in dit proefschrift helpt ontwikkelaars zich te richten op het modelleren van de interactiemogelijkheden van intelligente objecten en het uitzoeken van de mogelijke verbindingen tussen de objecten, wat het vergemakkelijkt om intelligente objecten te ontwikkelen en mogelijk maakt de apparaatinteroperabiliteit op niveau van semantiek te benaderen. 

In plaats van het beschrijven van een invoergebeurtenis via de hardware, die op een laag niveau een actie in werking zet, zijn de interactiegebeurtenissen in de ontologie omschreven als input actie op een hoger abstractie niveau waarbij de intentie van de gebruiker direct gerapporteerd wordt. Dit maakt het mogelijk voor ontwikkelaars om software te schrijven die reageert op de input acties op hoog niveau, zonder ondersteuning te vereisen voor alle mogelijke typen apparaten die de actie gegenereerd zouden kunnen hebben. De hi\"erarchie van gebeurtenissen kan automatisch worden afgeleid met behulp van semantische beredenering.

De software architectuur implementeert het zogenaamde publish/subscribe communicatie model, waarbij intelligente objecten veranderingen in data kunnen aanmelden, welke is weergegeven in triple formaat, en een notificatie ontvangen, iedere keer dat de triples zijn bijgewerkt, toegevoegd of verwijderd. De semantische beredenering wordt uitgevoerd door een informatieonderhandelaar (information broker), wat de implementatie voor de intelligente objecten vergemakkelijkt. 

Een testopstelling, samengesteld uit ongelijksoortige intelligente objecten ontwikkeld door verscheidene bedrijven and instituten, is gebruikt voor validatie van het ontwerp. De prestaties van de implementatie zijn ge\"evalueerd, met als resultaat een indicatie van een acceptabele reactietijd voor een genetwerkte gebruikersinterface. Een gebruikersanalyse van de ontologie en systeemimplementatie is uitgevoerd met ontwikkelaars, waarbij een vragenlijst is gebruikt die gebaseerd is op een bestaand bruikbaarheidsraamwerk. Uiteenlopende ontwerp patronen voor ontologie\"en zijn ge\"identificeerd tijdens ontwerpproces, en zijn gedocumenteerd in het proefschrift.
De resulterende uitkomst is een ontologie voor gebruikersinteractie met apparaten in een intelligente omgeving, waar apparaten de mogelijkheid hebben om interactiegebeurtenissen te delen en gebruik te maken van elkaars functionaliteit. 



\endgroup			

\vfill