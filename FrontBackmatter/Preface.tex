\cleardoublepage
\pagestyle{empty}

%*******************************************************
% Abstract
%*******************************************************
%\renewcommand{\abstractname}{Abstract}
\pdfbookmark[1]{Preface}{Preface}
\begingroup
\let\clearpage\relax
\let\cleardoublepage\relax
\let\cleardoublepage\relax

%thimbleby
\chapter*{Preface}

The work in this thesis was completed in close collaboration with another PhD candidate, Bram van der Vlist, whose thesis \cite{Bram} describes the more designer-related aspects in greater detail, whereas this thesis tends to focus on the more technical aspects of the work. Some overlap between the two theses is unavoidable, but we tried to keep this to a minimum. In particular, the design iterations described in Chapters \ref{DesignIteration1}, \ref{DesignIteration2} and \ref{DesignIteration3} were a combined effort, as well as the creation of a theory of semantic connections described in Chapter \ref{SemanticConnectionsTheory}. The device capability modelling (Chapter \ref{DeviceCapabilityModelling}) and event modelling (Chapter \ref{EventModelling}) techniques, as well as the work on ontology engineering (Chapter \ref{OntologyEngineering}), the extension of the ADK-SIB (Chapter \ref{SoftwareArchitecture}) and the evaluations (Chapter \ref{Evaluation}) are considered to be exclusive contributions of the author. 

The first person plural style of writing in the thesis is used to improve coherence and readability. Code fragments in the thesis are included either as explanation of an implementation or for the sake of clarity. The source code of the developed software\footnote{https://bitbucket.org/gniezen/semanticconnections} and ontologies\footnote{https://github.com/gniezen/ontologies} are available online.

A number of conference and journal papers related to this research were published in peer-reviewed proceedings and are listed on the next page. 


\endgroup			

\vfill