\cleardoublepage
\pagenumbering{gobble}
\pagestyle{empty}

%*******************************************************
% Abstract
%*******************************************************
%\renewcommand{\abstractname}{Abstract}
\pdfbookmark[1]{Abstract}{Abstract}
\begingroup


\chapter*{Abstract}

\begingroup
    \spacedallcaps{\myTitle} \\ 
\endgroup

\begingroup
\noindent
\spacedlowsmallcaps{\mySubtitle}\\ \bigskip
\endgroup

The thesis describes the design and development of an ontology and software framework to support user interaction in ubiquitous computing scenarios. An iterative approach was followed during the design process, with three design iterations documented in the thesis. The work was done in close cooperation with designers and other project partners, in order to elicit requirements and maintain a more holistic view of the various application areas.
 
The thesis describes an interaction model that shows the various concepts that are involved in user interaction in a smart space, including how these concepts work together. Based in the interaction model, a theory of semantic connections is introduced that focuses on the meaning of the connections between the different entities in a smart environment.

Ontologies are formal representations of concepts in a domain of interest and the relationships between these concepts. They are used to enable the exchange of information without requiring up-front standardisation. The ontology described in the thesis helps developers to focus on modelling the interaction capabilities of smart objects and inferring the possible connections between these objects, making it easier to build smart objects and enable device interoperability on a semantic level.

The software architecture implements the publish/subscribe messaging paradigm, enabling smart objects to subscribe to changes in data, represented in triple form, and be notified every time these triples are updated, added or removed. Semantic reasoning is performed on an information broker, simplifying the implementation on the smart objects.

A pilot deployment, composed of heterogeneous smart objects designed and manufactured by a range of companies and institutions, was used to validate the design. A performance evaluation was performed, where the results indicated acceptable response times for a networked user interface. A usability analysis of the ontology and system implementation was performed using a developer questionnaire based on an existing usability framework. Various ontology design patterns were identified during the course of the design, and are documented in the thesis.

The resulting design artefact is an ontology for user interaction in a smart environment, where devices are able to share interaction events and make use of each other's functionality.